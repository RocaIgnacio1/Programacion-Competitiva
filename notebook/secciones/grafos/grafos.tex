\newpage
\section{Grafos}

\subsection{Recorrer Grafos}
Dado un Grafo como lista de adjacencias

\begin{lstlisting}[language=C++]
#include <bits/stdc++.h>
#define MAXN 100000
using namespace std;

vector<int> G[MAXN];
bool visited[MAXN];
\end{lstlisting}

Podemos recorrerlo con DFS o con BFS.

\subsubsection{DFS}
\lstinputlisting[language=C++]{secciones/grafos/dfs.cpp}

\subsubsection{BFS}
\lstinputlisting[language=C++]{secciones/grafos/bfs.cpp}
\newpage

\subsection{Camino Mínimo}

\subsubsection{Bellman-Ford}
El algoritmo de Bellman-Ford encuentra el camino desde un nodo de origen a todos los nodos del grafo. \\
\textbf{Complejidad = O(nm)}

\lstinputlisting[language=C++]{secciones/grafos/bellman-ford.cpp}
\subsubsection{Ciclos negativos}
El algoritmo es capaz de detectar ciclos negativos. Para eso Se debe correr una vez m'as
\newpage

\subsubsection{Dijkstra}
El algorimo necesita que todos los pesos sean > 0 \\
\textbf{Complejidad = O(n + m log m)}
\lstinputlisting[language=C++]{secciones/grafos/dijkstra.cpp}
\newpage
\subsubsection{Floyd-Warshall}
\lstinputlisting[language=C++]{secciones/grafos/floyd.cpp}

\newpage
