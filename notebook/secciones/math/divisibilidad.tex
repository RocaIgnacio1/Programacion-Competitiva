\begin{center}
\tablefirsthead{\hline \textbf{Nro} & \textbf{Regla} & \textbf{Ejemplo} \\ \hline}
\tablehead{\hline \multicolumn{3}{|c|}{\tiny Contin\'ua} \\ \hline \textbf{Nro} & \textbf{Regla} & \textbf{Ejemplo} \\ }
% \tabletail{\hline}
% \tablelasttail{\hline}
\footnotesize
{
  \begin{xtabular}{|L{.1\columnwidth}|C{.4\columnwidth}|L{.4\columnwidth}|}
    1 & Todos los números & 5: porque si divides 5:1=5 y ese número es un múltiplo o divisor de cualquier número. \\ \hline
    2 & El número termina en una cifra par. & 378: porque la última cifra (8) es par. \\ \hline
    3 & La suma de sus cifras es un múltiplo de 3. & 480: porque 4+8+0=12 es múltiplo de 3. \\ \hline
    4 & Sus últimos dos dígitos son 0 o un múltiplo de 4. & 300 y 516 son divisibles entre 4 porque terminan en 00 y en 16, respectivamente, siendo este último un múltiplo de 4 (16=4*4). \\ \hline
    5 & La última cifra es 0 o 5. & 485: porque termina en 5. \\ \hline
    7 & Un número es divisible entre 7 cuando, al separar la última cifra de la derecha, multiplicarla por 2 y restarla de las cifras restantes la diferencia es igual a 0 o es un múltiplo de 7. Otro sistema: Si la suma de la multiplicación de los números por la serie 2,3,1,-2,-3,-1... da 0 o un múltiplo de 7. & 34349: separamos el 9, y lo duplicamos (18), entonces 3434-18=3416. Repetimos el proceso separando el 6 (341'6) y duplicándolo (12), entonces 341-12=329, y de nuevo, 32'9, 9*2=18, entonces 32-18=14; por lo tanto, 34349 es divisible entre 7 porque 14 es múltiplo de 7. Ejemplo método 2: 34349: [(2*3)+(3*4)+(1*3)-(2*4)-(3*9)]= 6+12+3-8-27 = -14.8 \\ \hline
    8 & Para saber si un número es divisible entre 8 hay que comprobar que sus tres últimas cifras sean divisibles entre 8. Si sus tres últimas cifras son divisibles entre 8 entonces el número también es divisible entre 8. & Ejemplo: El número 571.328 es divisible por 8 ya que sus últimas tres cifras (328) son divisibles por 8 (32 = 8*4 y 8 = 8*1). Realizando la división comprobamos que 571.328 : 8 = 71.416 \\ \hline
    9 & Un número es divisible por 9 cuando al sumar todas sus cifras el resultado es múltiplo de 9. & 504: sumamos 5+0+4=9 y como 9 es múltiplo de 9 504 es divisible por 9 5346: sumamos 5+3+4+6=18 y como 18 es múltiplo de 9, 5346 es divisible por 9. \\ \hline
    10 & La última cifra es 0. & 4680: porque termina en 0 \\ \hline
    11 & Sumando las cifras (del número) en posición impar por un lado y las de posición par por otro. Luego se resta el resultado de ambas sumas obtenidas. Si el resultado es cero o un múltiplo de 11, el número es divisible entre este. Si el número tiene solo dos cifras y estas son iguales será múltiplo de 11. & 42702: 4+7+2=13 · 2+0=2 · 13-2=11 → 42702 es múltiplo de 11. 66: porque las dos cifras son iguales. Entonces 66 es múltiplo de 11. \\ \hline
    13 & Un número es divisible entre 13 cuando, al separar la última cifra de la derecha, multiplicarla por 9 y restarla de las cifras restantes la diferencia es igual a 0 o es un múltiplo de 13 & 3822: separamos el último dos (382'2) y lo multiplicamos por 9, 2×9=18, entonces 382-18=364. Repetimos el proceso separando el 4 (36'4) y multiplicándolo por 9, 4×9=36, entonces 36-36=0; por lo tanto, 3822 es divisible entre 13. \\ \hline
    17 & Un número es divisible entre 17 cuando, al separar la última cifra de la derecha, multiplicarla por 5 y restarla de las cifras restantes la diferencia es igual a 0 o es un múltiplo de 17 & 2142: porque 214'2, 2*5=10, entonces 214-10=204, de nuevo, 20'4, 4*5=20, entonces 20-20=0; por lo tanto, 2142 es divisible entre 17. \\ \hline
    19 & Un número es divisible entre 19 si al separar la cifra de las unidades, multiplicarla por 2 y sumar a las cifras restantes el resultado es múltiplo de 19. & 3401: separamos el 1, lo doblamos (2) y sumamos 340+2= 342, ahora separamos el 2, lo doblamos (4) y sumamos 34+4=38 que es múltiplo de 19, luego 3401 también lo es. \\ \hline
    20 & Un número es divisible entre 20 si sus dos últimas cifras son ceros o múltiplos de 20. Cualquier número par que tenga uno o más ceros a la derecha, es múltiplo de 20. & 57860: Sus 2 últimas cifras son 60 (Que es divisible entre 20), por lo tanto 57860 es divisible entre 20. \\ \hline
    23 & Un número es divisible entre 23 si al separar la cifra de las unidades, multiplicar por 7 y sumar las cifras restantes el resultado es múltiplo de 23. & 253: separamos el 3, lo multiplicamos por 7 y sumamos 25+21= 46, 46 es múltiplo de 23 así que es divisible entre 23. \\ \hline
    25 & Un número es divisible entre 25 si sus dos últimas cifras son 00, o en múltiplo de 25 (25,50,75,...) & 650: Es múltiplo de 25 por lo cual es divisible. 400 también será divisible entre 25. \\ \hline
    29 & Un número es divisible entre 29 cuando, al separar la última cifra de la derecha, multiplicarla por 3 y restarla de las cifras restantes la diferencia es igual a 0 o es un múltiplo de 29 & 2436: separamos el 6 (243'6) y lo multiplicamos por 3, 6×3=18, entonces 243-18=225. Repetimos el proceso separando el 5 (22'5) y multiplicándolo por 3, 5×3=15, entonces 22-15=7, que no es divisible entre 29. \\ \hline
  \end{xtabular}
}
\end{center}
